\chapter{Introduction}

"In everyday life, a string—such as a shoelace—is usually used to secure something or hold it in place. When we tie a knot, the purpose is to help the string do its job. All too often, we run into a complicated and tangled mess of string, but ordinarily this happens by mistake." -Edward Witten on Knots and Quantum Theory

"... finding himself unable to untie the knot, the ends of which were secretly twisted round and folded up within it, cut it asunder with his sword." -Plutarch on the Legend of the Gordian Knot


Often in mathematics, the simplest objects are the most difficult to describe. At the very least, this is certainly true for mathematical knots. For centuries have mathematicians and others been puzzled by questions contained in what may be called knot theory. One of the guiding themes in knot theory is the search for new knot invariants, mechanisms defined combinatorially or otherwise, which partition the set of knots into equivalence classes so that each class is distinguishable from the others. 

The latter part of the twentieth century saw a surge of interest in this area with the discovery of the Jones polynomial \cite{Jon85}, a knot invariant which may be calculated by applications of a simple, recursive, diagrammatic formulas on knot diagrams \cite{Kau87}. This caused much excitement for many different reasons. Firstly, applications of the Jones polynomial were used to prove some of the Tait conjectures \cite{Kau87} \cite{Thi88} which were century-old open problems in knot theory. Separate from this, connections to physics were discovered. A knot may be regarded as the orbit of a charged particle in three-dimensional spacetime. It was realized that the Jones polynomial of a knot is, in a certain physical model, a quantum averaging of the Wilson operator, which is a ``probability amplitude" that the particle traveled along the orbit \cite{Wit89}. The Jones polynomial is also intimately connected to the representation theory, where it may be calculated as the quantum trace of an endomorpism, considered as a braid, of the two-dimensional representation of the quantum group $U_q(\mathfrak{sl}_2)$. There is also a connection to hyperbolic geometry, which has been formulated via the relatively long-standing (generalized) volume conjecture which states that there is a relationship between the complete hyperbolic volume of the complement of a knot in the $3$-sphere and its Jones polynomial \cite{Hik07}. The conjecture has been verified for certain special cases \cite{KT00}. More recently, the Jones polynomial has been categorified via Khovanov homology \cite{Kho00} where a link diagram gives rise to a chain complex whose homology is isotopy invariant and whose Euler characteristic is the Jones polynomial of the link. Khovanov homology is important for the physical interpretation because it allows for a knot to be interpreted as a physical object in four-dimensional spacetime. It is equally important for mathematics as categorification spawns additional structure which can be used to prove things, such as how Khavonov homology, considered as a knot invariant, can detect the unknot \cite{KM11}, where the analagous question for the Jones polynomial remains open.

A key fact about the diagrammatic formulas, called skein relations, which define the Jones polynomial is that they are defined locally at the crossings of the diagrams. Consequently, one may use the skein relations to define invariants of knots in more general $3$-manifolds $M$. The result is a family of modules, called skein modules, each of which are the set of linear combinations of links in $M$ modulo the skein relations \cite{Prz91}\AP{They were introduced by Józef H. Przytycki in 1987 (independently by Vladimir Turaev in 1988)}. The direction of this sort of generalization is orthogonal to the direction of categorification, although there is effort going into combining the two ideas \cite{APS04}. The skein module construction is functorial with respect to smooth orientation-preserving embeddings of $3$-manifolds, so the skein relations provide a sort of algebraic topology of smooth embedded submanifolds. Furthermore, the skein modules of a trivial interval-bundle over an oriented surface carries the structure of an algebra, hence they are called skein algebras. The skein algebras are of particular interest


Shortly after Jones' discovery of the Jones polynomial, mathematicians quickly defined more general knot polynomials. Where the Jones polynomial may be seen as corresponding to $\mathfrak{sl}_2$, the HOMFLYPT polynomial \cite{FHLMOY85} \cite{PT88} corresponds to $\mathfrak{sl}_n$ for any $n$, and the Dubrovnik polynomial \cite{Kau90} corresponds to $\mathfrak{sp}_{2n}$ or $\mathfrak{so}_n$ for any $n$. Similar to the Jones polynomial, each of these invariants may be defined and computed using similar recursive linear formulas on knot diagrams. These types of formulas, called skein relations, were first considered by Alexander \cite{Ale28} and popularized by Conway \cite{Con70} in the context of the Alexander polynomial. 