\chapter{The Kauffman Skein Algebra of the Torus}

\AP{Add some introduction and remark about collaboration.}


\section{Power Sum Elements}

Recall that there is a injective algebra homomorphism $\Lambda \to \ch(A)^+$ which sends the Schur function $s_\lambda$ to the minimal idempotent closure $Q_\lambda$. Use $h_n := Q_{(n)}$ to denote the image of the $n^\textrm{th}$ complete homogeneous symmetric function under this homomorphism. In \AP{MS} the authors import power sum elements from $\Lambda$ to $P_k \in \ch(A)$. The power sum elements have a concrete definition in $\Lambda$, but alternatively they may be defined using an equation of formal power series in the ring $\ch(A)[[t]]$ as
\begin{equation}
\sum_{k=1}^\infty \frac{P_k}{k} t^k = \ln \Bigg( 1 + \sum_{n=1}^\infty h_n t^n \Bigg)
\end{equation}
which writes each $P_k$ in terms of the generators $h_n$. 

Using the Beliakova-Blanchet section $\Gamma: H_n \to BMW_n$, we may emulate this definition to define ``power sum" elements $\tilde{P}_k \in \cd(A)$ by the formal power series equation
\begin{equation}
\sum_{k=1}^\infty \frac{\tilde{P}_k}{k} t^k = \ln \Bigg( 1 + \sum_{n=1}^\infty \tilde{h}_n t^n \Bigg)
\end{equation}
where $\tilde{h}_n := \tilde{Q}_{(n)}$ is the annular closure of the BMW symmetrizers $\tilde{y}_n = \Gamma(y_n)$.


\subsection{Commutation Relations} \label{sec:commurationrelations}

Here we will continue our discussion of Section \ref{sec:relativeannulus}. In particular, we will prove the following Theorem.

\begin{theorem} \label{thm:powersumcommutator}
For any $k \geq 1$, the relation
\[
e \cdot \tilde{P}_k - \tilde{P}_k \cdot e = (s^k - s^{-k}) (a^k - a^{-k})
\]
\end{theorem}

We will split the proof of this theorem into two technical lemmas.

\begin{lemma}
The relations of Theorem \ref{thm:powersumcommutator} hold if and only if 
\begin{equation}
e \cdot (\tilde{h}_{n+2} + \tilde{h}_n) - (\tilde{h}_{n+2} + \tilde{h}_n) \cdot e = (sa + s^{-1}a^{-1}) (e \cdot \tilde{h}_{n+1}) - (s^{-1}a + sa^{-1}) (\tilde{h}_{n+1} \cdot e)
\end{equation}
for all $n \geq -1$ where $\tilde{h}_0 := 1$ and $\tilde{h}_{-1} := 0$. 
\end{lemma}
\begin{proof}
The relations of Theorem \ref{thm:powersumcommutator} may be organized into a single power series equation
\begin{equation}
\sum_{k=1}^\infty \frac{e \cdot \tilde{P}_k - \tilde{P}_k \cdot e}{k} t^k = \sum_{k=1}^{\infty} \frac {(s^k - s^{-k}) (a^k - a^{-k})}{k} t^k
\end{equation}
in $\ca_\cd [[t]]$. Rewrite this equation as
\begin{equation*}
e \cdot \Bigg( \sum_{k=1}^\infty \tilde{P}_k \Bigg) - \Bigg( \sum_{k=1}^\infty \tilde{P}_k \Bigg) \cdot e = \sum_{k=1}^{\infty} \frac {(sat)^k}{k} + \sum_{k=1}^{\infty} \frac {(s^{-1}a^{-1}t)^k}{k} - \sum_{k=1}^{\infty} \frac {(s^{-1}at)^k}{k} - \sum_{k=1}^{\infty} \frac {(sa^{-1}t)^k}{k}
\end{equation*}
For shorthand, define 
\[
H(t) := 1 + \sum_{n=1}^\infty \tilde{h}_n t^n
\]
and recall the Taylor series expansion
\[
-\ln(1-x) = \sum_{k=1}^\infty \frac{x^k}{k}
\]
which is a variation of the Newton-Mercator series. 
\end{proof}


\section{All Relations}




\section{Perpendicular Relations}




\section{Main Theorem}




\section{Compatibility With the Kauffman Bracket Skein Algebra of the Torus}