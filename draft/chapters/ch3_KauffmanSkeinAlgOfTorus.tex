\chapter{The Kauffman Skein Algebra of the Torus}

\AP{Add some introduction and remark about collaboration.}


\section{Power Sum Elements}

Recall that there is a injective algebra homomorphism $\Lambda \to \ch(A)^+$ which sends the Schur function $s_\lambda$ to the minimal idempotent closure $Q_\lambda$. Use $h_n := Q_{(n)}$ to denote the image of the $n^\textrm{th}$ complete homogeneous symmetric function under this homomorphism. In \AP{MS} the authors import power sum elements from $\Lambda$ to $P_k \in \ch(A)$. The power sum elements have a concrete definition in $\Lambda$, but alternatively they may be defined using an equation of formal power series in the ring $\ch(A)[[t]]$ as
\begin{equation}
\sum_{k=1}^\infty \frac{P_k}{k} t^k = \ln \Bigg( 1 + \sum_{n=1}^\infty h_n t^n \Bigg)
\end{equation}
which writes each $P_k$ in terms of the generators $h_n$. 

Using the Beliakova-Blanchet section $\Gamma: H_n \to BMW_n$, we may emulate this definition to define ``power sum" elements $\tilde{P}_k \in \cd(A)$ by the formal power series equation
\begin{equation}
\sum_{k=1}^\infty \frac{\tilde{P}_k}{k} t^k = \ln \Bigg( 1 + \sum_{n=1}^\infty \tilde{h}_n t^n \Bigg)
\end{equation}
where $\tilde{h}_n := \tilde{Q}_{(n)}$ is the annular closure of the BMW symmetrizers $\tilde{y}_n = \Gamma(y_n)$.


\subsection{Commutation Relations} \label{sec:commurationrelations}

Here we will continue our discussion of Section \ref{sec:relativeannulus}. In particular, we will prove the following Theorem.

\begin{theorem} \label{thm:powersumcommutator}
For any $k \geq 1$, the relation
\[
e \cdot \tilde{P}_k - \tilde{P}_k \cdot e = (s^k - s^{-k}) (a^k - a^{-k})
\]
\end{theorem}

We will split the proof of this theorem into two technical lemmas.

\begin{lemma} \label{lem:powersumcommutator1}
The relations of Theorem \ref{thm:powersumcommutator} hold if and only if 
\begin{equation} \label{eq:powersumcommutator1}
e \cdot (\tilde{h}_{n+2} + \tilde{h}_n) - (\tilde{h}_{n+2} + \tilde{h}_n) \cdot e = (sa + s^{-1}a^{-1}) (e \cdot \tilde{h}_{n+1}) - (s^{-1}a + sa^{-1}) (\tilde{h}_{n+1} \cdot e)
\end{equation}
for all $n \geq -1$ where $\tilde{h}_0 := 1$ and $\tilde{h}_{i} := 0$ for any $i \leq -1$. 
\end{lemma}
\begin{proof}
The relations of Theorem \ref{thm:powersumcommutator} may be organized into a single power series equation
\begin{equation} \label{eq:powersumcommutator1}
\sum_{k=1}^\infty \frac{e \cdot \tilde{P}_k - \tilde{P}_k \cdot e}{k} t^k = \sum_{k=1}^{\infty} \frac {(s^k - s^{-k}) (a^k - a^{-k})}{k} t^k
\end{equation}
in $\ca_\cd [[t]]$. Rewrite this equation as
\begin{equation} \label{eq:powersumcommutator2} 
e \cdot \Bigg( \sum_{k=1}^\infty \tilde{P}_k \Bigg) - \Bigg( \sum_{k=1}^\infty \tilde{P}_k \Bigg) \cdot e = \sum_{k=1}^{\infty} \frac {(sat)^k}{k} + \sum_{k=1}^{\infty} \frac {(s^{-1}a^{-1}t)^k}{k} - \sum_{k=1}^{\infty} \frac {(s^{-1}at)^k}{k} - \sum_{k=1}^{\infty} \frac {(sa^{-1}t)^k}{k}
\end{equation}
We can make sense of the left-hand side by extending the algebra homomorphism $x \mapsto e \cdot (-)$ to an algebra homomorphism of rings of formal power series
\begin{center}
\begin{tikzcd}
\cd(A) \arrow[r, "e \cdot (-)"] \arrow[d, hook] & \ca_\cd \arrow[d, hook] \\
\cd(A)[[t]] \arrow[r, "e \cdot (-)"] & \ca_\cd[[t]]
\end{tikzcd}
\end{center}
and similarly for $(-) \cdot e$. Now for shorthand, define 
\[
H(t) := 1 + \sum_{n=1}^\infty \tilde{h}_n t^n
\]
and recall the Taylor series expansion
\[
-\ln(1-x) = \sum_{k=1}^\infty \frac{x^k}{k}
\]
which is a variation of the Newton-Mercator series. Then the Equation \eqref{eq:powersumcommutator1} becomes
\begin{equation} \label{eq:powersumcommutator3} 
e \cdot \Big( \ln\big(H(t)\big) \Big) - \big( \ln \big(H(t)\big) \Big) \cdot e = - \ln(1 - sat) - \ln(1 - s^{-1}a^{-1}t) + \ln(1 - s^{-1}at) + \ln(1 - sa^{-1}t).
\end{equation}
The maps $e \cdot (-)$ and $(-) \cdot e$ commute with the natural logarithm. Use this and other natural log properties to write
\begin{equation}
\ln\Big( e \cdot \big( H(t) \big) \big( 1 - (sa + s^{-1}a^{-1})t + t^2 \big) \Big) = \ln\Big( \big( H(t) \big) \cdot e \big( 1 - (s^{-1}a + sa^{-1})t + t^2 \big) \Big).
\end{equation}
Exponentiating both sides and equating coefficients gives the system of equations defined in the statement of the lemma. Each step of the proof is invertible, and thus the two sets of relations are logically equivalent.
\end{proof}

\begin{lemma}
The relations of Lemma \ref{lem:powersumcommutator1} hold.
\end{lemma}
\begin{proof}
If $n=-1$, the relation we would like to show becomes 
\begin{equation*}
e \cdot \tilde{h}_1 - \tilde{h}_1 \cdot e = \{1\} \left( a - a^{-1} \right)
\end{equation*}
which is just the Dubrovnik skein relation. 

For general values of $n$, the proof is a technical computation using repeated applications of the recursive formula for the $\tilde{h}_n$. Since we will need them here, let's recall the formulas given in Section \ref{sec:relativeannulus}. There are recursive formulas
\begin{align}
[n+1] W_n &= e \cdot \tilde{h}_n + [n] s^{-1} a W_{n-1} + [n] s^{-1} \beta_n a^{-1} W^*_{n-1}, \label{eq:r1} \\
[n+1] W^*_n &= \tilde{h}_n \cdot e + [n] s^{-1} a W^*_{n-1} + [n] s^{-1} \beta_n a W_{n-1}, \label{eq:r2} \\
[n+1] W_n &= \tilde{h}_n \cdot e + [n] s a W_{n-1} + [n] s \bar{\beta}_n a^{-1} W^*_{n-1}, \label{eq:r3} \\
[n+1] W^*_n &= e \cdot \tilde{h}_n + [n] s a^{-1} W^*_{n-1} + [n] s^{-1} \bar{\beta}_n a W_{n-1} \label{eq:r4}
\end{align}
and the ``fundamental relation"
\begin{equation}
\tilde{h}_n \cdot e - e \cdot \tilde{h}_n = \{n\} (a^{-1} W^*_{n-1} - a W_{n-1}). \label{eq:f}
\end{equation}

Start by applying Equation \eqref{eq:f} to the relation of Lemma \ref{lem:powersumcommutator1} to obtain an equivalent relation
\begin{align}\label{eq_perprel4}
\begin{split}
\{n+2\} \left( aW_{n+1} - a^{-1}W^*_{n+1} \right) =& \left( sa+s^{-1}a^{-1} \right) \left( e \cdot \tilde{h}_{n+1} \right) - \left( sa^{-1}+s^{-1}a \right) \left( \tilde{h}_{n+1} \cdot e \right) \\
 & \qquad\qquad\qquad\qquad\qquad\qquad - \{n\}\left( aW_{n-1}-a^{-1}W^*_{n-1} \right).
\end{split}
\end{align}
We will show that the left-hand side of this equation may be reduced to the right-hand side by a series of applications of the recursive formulas, which we will signify with an asterisk $\ast$. 
\begin{eqnarray*}
&& \{n+2\} \left( aW_{n+1} - a^{-1}W^*_{n+1} \right) \\
\overset{\ast}{=}&& \{n+2\} \left( \frac{a}{[n+2]} \left( e \cdot \tilde{h}_{n+1} + [n+1]s^{-1}aW_n + [n+1]s^{-1}\beta_{n+1}a^{-1}W^*_n \right) \right.- \\
&&\left.\qquad\qquad\frac{a^{-1}}{[n+2]} \left( \tilde{h}_{n+1} \cdot e + [n+1]s^{-1}a^{-1}W^*_n + [n+1]s^{-1}\beta_{n+1}aW_n \right) \right)  \\
=&& \left( s - s^{-1} \right) \left( \left( a \cdot \tilde{h}_{n+1} + [n+1]s^{-1}a^2W_n + [n+1]s^{-1}\beta_{n+1}W^*_n \right) \right. \\
&&\qquad\qquad\,\,\,\,\,\left.-\left( \tilde{h}_{n+1} \cdot a^{-1} + [n+1]s^{-1}a^{-2}W^*_n + [n+1]s^{-1}\beta_{n+1}W_n \right)\right) \\
\overset{\ast}{=}&&\left( s-s^{-1} \right) \left( \left( a \cdot \tilde{h}_{n+1} + s^{-2}a \left( [n+2]W_{n+1} - \tilde{h}_{n+1} \cdot e - [n+1]s\bar{\beta}_{n+1}a^{-1}W^*_n \right) \right. \right. \\
&&\qquad\qquad\qquad\qquad\qquad\qquad\qquad\qquad\qquad\qquad\qquad\qquad
+ [n+1]s^{-1}\beta_{n+1}W^*_n \Big)  \\
&&\qquad\qquad\,\,\,\, \left. - \left( \tilde{h}_{n+1} \cdot a^{-1} + s^{-2}a^{-1}\left( [n+2]W^*_{n+1} - e \cdot \tilde{h}_{n+1} - [n+1]s\bar{\beta}_{n+1}aW_n \right) \right. \right. \\
&&\qquad\qquad\qquad\qquad\qquad\qquad\qquad\qquad\qquad\qquad\qquad\qquad\quad\quad
+ [n+1]s^{-1}\beta_{n+1}W_n \Big) \Big) \\
=&&\left( sa + s^{-1}a^{-1} \right) \left( e \cdot \tilde{h}_{n+1} \right) - \left( sa^{-1} + s^{-1}a \right) \left( \tilde{h}_{n+1} \cdot e \right) \\
&+& \left( s^{-1}a^{-1} + s^{-3}a \right) \left( \tilde{h}_{n+1} \cdot e \right) - \left( s^{-1}a + s^{-3}a^{-1} \right) \left( e \cdot \tilde{h}_{n+1} \right) \\
&+& \{n+2\}s^{-2}\left( aW_{n+1} - a^{-1} W^*_{n+1} \right) + \{n+1\}s^{-1}\left( \bar{\beta}_{n+1} - \beta_{n+1} \right) \left( W_n - W^*_n \right).
\end{eqnarray*}

We break the computation here to note that the first two terms in the last line also appear on the right hand side of \eqref{eq_perprel4}. Thus, we would like to prove the following equality:
\begin{eqnarray*}
&-&\{n\} \left( aW_{n-1} - a^{-1}W^*_{n-1} \right) \\
=&& \left( s^{-1}a^{-1} + s^{-3}a \right) \left( \tilde{h}_{n+1} \cdot e \right) - \left( s^{-1}a + s^{-3}a^{-1} \right) \left( e \cdot \tilde{h}_{n+1} \right) \\
&+&\{n+2\}s^{-2}\left( aW_{n+1} - a^{-1} W^*_{n+1} \right) - \{n+1\}s^{-1}\left( \bar{\beta}_{n+1} - \beta_{n+1} \right) \left( W_n - W^*_n \right).
\end{eqnarray*}
We will work the right-hand side of this above equation down to the left-hand side by continuing to apply the same identities. A large number of terms cancel and what remains is the desired relation.
%\pagebreak
\begin{eqnarray*}
&& \left( s^{-1}a^{-1} + s^{-3}a \right) \left( \tilde{h}_{n+1} \cdot e \right) - \left( s^{-1}a + s^{-3}a^{-1} \right) \left( e \cdot \tilde{h}_{n+1} \right) \\
&+& \{n+2\}s^{-2}\left( aW_{n+1} - a^{-1} W^*_{n+1} \right) - \{n+1\}s^{-1}\left( \bar{\beta}_{n+1} - \beta_{n+1} \right) \left( W_n - W^*_n \right) \\
=&& \left( s^{-1}a^{-1} + s^{-3}a \right) \left( \tilde{h}_{n+1} \cdot e \right) - \left( s^{-1}a+s^{-3}a^{-1} \right) \left( e \cdot \tilde{h}_{n+1} \right) \\
&+& [n+2]\left( s^{-1}-s^{-3} \right) \left(aW_{n+1} - a^{-1}W^*_{n+1} \right) \\
&+& [n+1]\left( 1-s^{-2} \right) \left( \bar{\beta}_{n+1}-\beta_{n+1} \right) \left(W_n - W^*_n \right) \\
=&& s^{-1}a\left( [n+2]W_{n+1} - e \cdot \tilde{h}_{n+1} \right) \\
&+& s^{-3}a^{-1}\left( [n+2]W^*_{n+1} - e \cdot \tilde{h}_{n+1} \right) \\
&-& s^{-3}a\left( [n+2]W_{n+1} - \tilde{h}_{n+1} \cdot e \right) \\
&-& s^{-1}a^{-1}\left( [n+2]W^*_{n+1} - \tilde{h}_{n+1} \cdot e \right) \\
&+& [n+1]\left( 1-s^{-2} \right) \left( \bar{\beta}_{n+1} -\beta_{n+1} \right) \left( W_n - W^*_n \right) \\
\overset{\ast}{=}&& s^{-1}a\left( [n+1]s^{-1}aW_n + [n+1]s^{-1}\beta_{n+1}a^{-1}W^*_n \right) \\
&+& s^{-3}a^{-1}\left( [n+1]sa^{-1}W^*_n + [n+1]s\bar{\beta}_{n+1}aW_n \right) \\
&-&s^{-3}a\left( [n+1]saW_n + [n+1]s\bar{\beta}_{n+1}a^{-1}W^*_n \right) \\
&-& s^{-1}a^{-1}\left( [n+1]s^{-1}a^{-1}W^*_n + [n+1]s^{-1}\beta_{n+1}aW_n \right) \\
&+& [n+1]\left(1-s^{-2} \right) \left( \bar{\beta}_{n+1} - \beta_{n+1} \right) \left( W_n - W^*_n \right)\\
=&& [n+1]\left( s^{-2}a^2W_n + s^{-2}\beta_{n+1}W^*_n + s^{-2}a^{-2}W^*_n + s^{-2}\bar{\beta}_{n+1}W_n - s^{-2}a^{2}W_n \right. \\
&&\qquad\quad \left. - s^{-2}\bar{\beta}_{n+1}W^*_n - s^{-2}a^{-2}W^*_n - s^{-2}\beta_{n+1}W_n + \bar{\beta}_{n+1}W_n - \bar{\beta}_{n+1}W^*_n - \beta_{n+1}W_n \right. \\
&&\qquad\quad \left. + \beta_{n+1}W^*_n - s^{-2}\bar{\beta}_{n+1}W_n + s^{-2}\bar{\beta}_{n+1}W^*_n + s^{-2}\beta_{n+1}W_n - s^{-2}\beta_{n+1}W^*_n \right) \\
=&& [n+1]\left( \bar{\beta}_{n+1} - \beta_{n+1} \right) \left( W_n - W^*_n \right) \\
=&& \left( \bar{\beta}_{n+1} - \beta_{n+1} \right) \left( \left( [n+1]W_n \right) - \left( [n+1]W^*_n \right) \right) \\
\overset{\ast}{=}&& \left( \bar{\beta}_{n+1} - \beta_{n+1} \right) \left( \left( e \cdot \tilde{h}_{n} + [n]s^{-1}aW_{n-1} + [n]s^{-1}\beta_{n}a^{-1}W^*_{n-1} \right) \right. \\
&& \qquad\qquad\qquad\qquad \left. -\left( e \cdot \tilde{h}_{n} + [n]sa^{-1}W^*_{n-1} + [n]s\bar{\beta}_{n}aW_{n-1} \right) \right) \\
=&& \left( \bar{\beta}_{n+1} - \beta_{n+1} \right) \left( [n]\left( s^{-1} - s\bar{\beta}_{n} \right) aW_{n-1} - [n]\left( s-s^{-1}\beta_{n} \right) a^{-1}W^*_{n-1} \right) \\
=&& [n]\left(\bar{\beta}_{n+1} - \beta_{n+1} \right) \left( s - s^{-1}\beta_{n} \right) \left( aW_{n-1} - a^{-1}W^*_{n-1} \right) \\
=&-& \{n\}\left( aW_{n-1} - a^{-1}W^*_{n-1} \right).
\end{eqnarray*}
Where the last equality follows from a quick computation in the base ring. This completes the proof.
\end{proof}


\section{All Relations}




\section{Perpendicular Relations}




\section{Main Theorem}




\section{Compatibility With the Kauffman Bracket Skein Algebra of the Torus}