\chapter{The Kauffman Skein Algebra of the Torus}

\AP{Add some introduction and remark about collaboration.}


\section{Power Sum Elements}

Recall that there is a injective algebra homomorphism $\Lambda \to \ch(A)^+$ which sends the Schur function $s_\lambda$ to the minimal idempotent closure $Q_\lambda$. Use $h_n := Q_{(n)}$ to denote the image of the $n^\textrm{th}$ complete homogeneous symmetric function under this homomorphism. In \AP{MS} the authors import power sum elements from $\Lambda$ to $P_k \in \ch(A)$. The power sum elements have a concrete definition in $\Lambda$, but alternatively they may be defined using an equation of formal power series in the ring $\ch(A)[[t]]$ as
\begin{equation}
\sum_{k=1}^\infty \frac{P_k}{k} t^k = \ln \Bigg( 1 + \sum_{n=1}^\infty h_n t^n \Bigg)
\end{equation}
which writes each $P_k$ in terms of the generators $h_n$. 

Using the Beliakova-Blanchet section $\Gamma: H_n \to BMW_n$, we may emulate this definition to define ``power sum" elements $\tilde{P}_k \in \cd(A)$ by the formal power series equation
\begin{equation}
\sum_{k=1}^\infty \frac{\tilde{P}_k}{k} t^k = \ln \Bigg( 1 + \sum_{n=1}^\infty \tilde{h}_n t^n \Bigg)
\end{equation}
where $\tilde{h}_n := \tilde{Q}_{(n)}$ is the annular closure of the BMW symmetrizers $\tilde{y}_n = \Gamma(y_n)$.


\subsection{Commutation Relations} \label{sec:commurationrelations}

Here we will continue our discussion of Section \ref{sec:relativeannulus}. In particular, we will prove the following Theorem.

\begin{theorem} \label{thm:powersumcommutator}
For any $k \geq 1$, the relation
\[
e \cdot \tilde{P}_k - \tilde{P}_k \cdot e = (s^k - s^{-k}) (a^k - a^{-k})
\]
holds. Equivalently,
\[
a^i \cdot \tilde{P}_k - \tilde{P}_k \cdot a^i = (s^k - s^{-k}) (a^{k+i} - a^{-k+i})
\]
\end{theorem}

We will split the proof of this theorem into two technical lemmas.

\begin{lemma} \label{lem:powersumcommutator1}
The relations of Theorem \ref{thm:powersumcommutator} hold if and only if 
\begin{equation} \label{eq:skewcommutator}
e \cdot (\tilde{h}_{n+2} + \tilde{h}_n) - (\tilde{h}_{n+2} + \tilde{h}_n) \cdot e = (sa + s^{-1}a^{-1}) (e \cdot \tilde{h}_{n+1}) - (s^{-1}a + sa^{-1}) (\tilde{h}_{n+1} \cdot e)
\end{equation}
for all $n \geq -1$ where $\tilde{h}_0 := 1$ and $\tilde{h}_{i} := 0$ for any $i \leq -1$. 
\end{lemma}
\begin{proof}
The relations of Theorem \ref{thm:powersumcommutator} may be organized into a single power series equation
\begin{equation} \label{eq:skewcommutator}
\sum_{k=1}^\infty \frac{e \cdot \tilde{P}_k - \tilde{P}_k \cdot e}{k} t^k = \sum_{k=1}^{\infty} \frac {(s^k - s^{-k}) (a^k - a^{-k})}{k} t^k
\end{equation}
in $\ca_\cd [[t]]$. Rewrite this equation as
\begin{equation} \label{eq:powersumcommutator2} 
e \cdot \Bigg( \sum_{k=1}^\infty \tilde{P}_k \Bigg) - \Bigg( \sum_{k=1}^\infty \tilde{P}_k \Bigg) \cdot e = \sum_{k=1}^{\infty} \frac {(sat)^k}{k} + \sum_{k=1}^{\infty} \frac {(s^{-1}a^{-1}t)^k}{k} - \sum_{k=1}^{\infty} \frac {(s^{-1}at)^k}{k} - \sum_{k=1}^{\infty} \frac {(sa^{-1}t)^k}{k}
\end{equation}
We can make sense of the left-hand side by extending the algebra homomorphism $x \mapsto e \cdot (-)$ to an algebra homomorphism of rings of formal power series
\begin{center}
\begin{tikzcd}
\cd(A) \arrow[r, "e \cdot (-)"] \arrow[d, hook] & \ca_\cd \arrow[d, hook] \\
\cd(A)[[t]] \arrow[r, "e \cdot (-)"] & \ca_\cd[[t]]
\end{tikzcd}
\end{center}
and similarly for $(-) \cdot e$. Now for shorthand, define 
\[
H(t) := 1 + \sum_{n=1}^\infty \tilde{h}_n t^n
\]
and recall the Taylor series expansion
\[
-\ln(1-x) = \sum_{k=1}^\infty \frac{x^k}{k}
\]
which is a variation of the Newton-Mercator series. Then the Equation \eqref{eq:skewcommutator} becomes
\begin{equation} \label{eq:powersumcommutator3} 
e \cdot \Big( \ln\big(H(t)\big) \Big) - \big( \ln \big(H(t)\big) \Big) \cdot e = - \ln(1 - sat) - \ln(1 - s^{-1}a^{-1}t) + \ln(1 - s^{-1}at) + \ln(1 - sa^{-1}t).
\end{equation}
The maps $e \cdot (-)$ and $(-) \cdot e$ commute with the natural logarithm. Use this and other natural log properties to write
\begin{equation}
\ln\Big( e \cdot \big( H(t) \big) \big( 1 - (sa + s^{-1}a^{-1})t + t^2 \big) \Big) = \ln\Big( \big( H(t) \big) \cdot e \big( 1 - (s^{-1}a + sa^{-1})t + t^2 \big) \Big).
\end{equation}
Exponentiating both sides and equating coefficients gives the system of equations defined in the statement of the lemma. Each step of the proof is invertible, and thus the two sets of relations are logically equivalent.
\end{proof}

\begin{lemma}
The relations of Lemma \ref{lem:powersumcommutator1} hold.
\end{lemma}
\begin{proof}
If $n=-1$, the relation we would like to show becomes 
\begin{equation*}
e \cdot \tilde{h}_1 - \tilde{h}_1 \cdot e = \{1\} \left( a - a^{-1} \right)
\end{equation*}
which is just the Dubrovnik skein relation. 

For general values of $n$, the proof is a technical computation using repeated applications of the recursive formula for the $\tilde{h}_n$. Since we will need them here, let's recall the formulas given in Section \ref{sec:relativeannulus}. There are recursive formulas
\begin{align}
[n+1] W_n &= e \cdot \tilde{h}_n + [n] s^{-1} a W_{n-1} + [n] s^{-1} \beta_n a^{-1} W^*_{n-1}, \label{eq:r1} \\
[n+1] W^*_n &= \tilde{h}_n \cdot e + [n] s^{-1} a W^*_{n-1} + [n] s^{-1} \beta_n a W_{n-1}, \label{eq:r2} \\
[n+1] W_n &= \tilde{h}_n \cdot e + [n] s a W_{n-1} + [n] s \bar{\beta}_n a^{-1} W^*_{n-1}, \label{eq:r3} \\
[n+1] W^*_n &= e \cdot \tilde{h}_n + [n] s a^{-1} W^*_{n-1} + [n] s^{-1} \bar{\beta}_n a W_{n-1} \label{eq:r4}
\end{align}
and the ``fundamental relation"
\begin{equation}
\tilde{h}_n \cdot e - e \cdot \tilde{h}_n = \{n\} (a^{-1} W^*_{n-1} - a W_{n-1}). \label{eq:f}
\end{equation}

Start by applying Equation \eqref{eq:f} to the relation of Lemma \ref{lem:powersumcommutator1} to obtain an equivalent relation
\begin{align}\label{eq_perprel4}
\begin{split}
\{n+2\} \left( aW_{n+1} - a^{-1}W^*_{n+1} \right) =& \left( sa+s^{-1}a^{-1} \right) \left( e \cdot \tilde{h}_{n+1} \right) - \left( sa^{-1}+s^{-1}a \right) \left( \tilde{h}_{n+1} \cdot e \right) \\
 & \qquad\qquad\qquad\qquad\qquad\qquad - \{n\}\left( aW_{n-1}-a^{-1}W^*_{n-1} \right).
\end{split}
\end{align}
We will show that the left-hand side of this equation may be reduced to the right-hand side by a series of applications of the recursive formulas, which we will signify with an asterisk $\ast$. 
\begin{eqnarray*}
&& \{n+2\} \left( aW_{n+1} - a^{-1}W^*_{n+1} \right) \\
\overset{\ast}{=}&& \{n+2\} \left( \frac{a}{[n+2]} \left( e \cdot \tilde{h}_{n+1} + [n+1]s^{-1}aW_n + [n+1]s^{-1}\beta_{n+1}a^{-1}W^*_n \right) \right.- \\
&&\left.\qquad\qquad\frac{a^{-1}}{[n+2]} \left( \tilde{h}_{n+1} \cdot e + [n+1]s^{-1}a^{-1}W^*_n + [n+1]s^{-1}\beta_{n+1}aW_n \right) \right)  \\
=&& \left( s - s^{-1} \right) \left( \left( a \cdot \tilde{h}_{n+1} + [n+1]s^{-1}a^2W_n + [n+1]s^{-1}\beta_{n+1}W^*_n \right) \right. \\
&&\qquad\qquad\,\,\,\,\,\left.-\left( \tilde{h}_{n+1} \cdot a^{-1} + [n+1]s^{-1}a^{-2}W^*_n + [n+1]s^{-1}\beta_{n+1}W_n \right)\right) \\
\overset{\ast}{=}&&\left( s-s^{-1} \right) \left( \left( a \cdot \tilde{h}_{n+1} + s^{-2}a \left( [n+2]W_{n+1} - \tilde{h}_{n+1} \cdot e - [n+1]s\bar{\beta}_{n+1}a^{-1}W^*_n \right) \right. \right. \\
&&\qquad\qquad\qquad\qquad\qquad\qquad\qquad\qquad\qquad\qquad\qquad\qquad
+ [n+1]s^{-1}\beta_{n+1}W^*_n \Big)  \\
&&\qquad\qquad\,\,\,\, \left. - \left( \tilde{h}_{n+1} \cdot a^{-1} + s^{-2}a^{-1}\left( [n+2]W^*_{n+1} - e \cdot \tilde{h}_{n+1} - [n+1]s\bar{\beta}_{n+1}aW_n \right) \right. \right. \\
&&\qquad\qquad\qquad\qquad\qquad\qquad\qquad\qquad\qquad\qquad\qquad\qquad\quad\quad
+ [n+1]s^{-1}\beta_{n+1}W_n \Big) \Big) \\
=&&\left( sa + s^{-1}a^{-1} \right) \left( e \cdot \tilde{h}_{n+1} \right) - \left( sa^{-1} + s^{-1}a \right) \left( \tilde{h}_{n+1} \cdot e \right) \\
&+& \left( s^{-1}a^{-1} + s^{-3}a \right) \left( \tilde{h}_{n+1} \cdot e \right) - \left( s^{-1}a + s^{-3}a^{-1} \right) \left( e \cdot \tilde{h}_{n+1} \right) \\
&+& \{n+2\}s^{-2}\left( aW_{n+1} - a^{-1} W^*_{n+1} \right) + \{n+1\}s^{-1}\left( \bar{\beta}_{n+1} - \beta_{n+1} \right) \left( W_n - W^*_n \right).
\end{eqnarray*}

We break the computation here to note that the first two terms in the last line also appear on the right hand side of \eqref{eq_perprel4}. Thus, we would like to prove the following equality:
\begin{eqnarray*}
&-&\{n\} \left( aW_{n-1} - a^{-1}W^*_{n-1} \right) \\
=&& \left( s^{-1}a^{-1} + s^{-3}a \right) \left( \tilde{h}_{n+1} \cdot e \right) - \left( s^{-1}a + s^{-3}a^{-1} \right) \left( e \cdot \tilde{h}_{n+1} \right) \\
&+&\{n+2\}s^{-2}\left( aW_{n+1} - a^{-1} W^*_{n+1} \right) - \{n+1\}s^{-1}\left( \bar{\beta}_{n+1} - \beta_{n+1} \right) \left( W_n - W^*_n \right).
\end{eqnarray*}
We will work the right-hand side of this above equation down to the left-hand side by continuing to apply the same identities. A large number of terms cancel and what remains is the desired relation.
%\pagebreak
\begin{eqnarray*}
&& \left( s^{-1}a^{-1} + s^{-3}a \right) \left( \tilde{h}_{n+1} \cdot e \right) - \left( s^{-1}a + s^{-3}a^{-1} \right) \left( e \cdot \tilde{h}_{n+1} \right) \\
&+& \{n+2\}s^{-2}\left( aW_{n+1} - a^{-1} W^*_{n+1} \right) - \{n+1\}s^{-1}\left( \bar{\beta}_{n+1} - \beta_{n+1} \right) \left( W_n - W^*_n \right) \\
=&& \left( s^{-1}a^{-1} + s^{-3}a \right) \left( \tilde{h}_{n+1} \cdot e \right) - \left( s^{-1}a+s^{-3}a^{-1} \right) \left( e \cdot \tilde{h}_{n+1} \right) \\
&+& [n+2]\left( s^{-1}-s^{-3} \right) \left(aW_{n+1} - a^{-1}W^*_{n+1} \right) \\
&+& [n+1]\left( 1-s^{-2} \right) \left( \bar{\beta}_{n+1}-\beta_{n+1} \right) \left(W_n - W^*_n \right) \\
=&& s^{-1}a\left( [n+2]W_{n+1} - e \cdot \tilde{h}_{n+1} \right) \\
&+& s^{-3}a^{-1}\left( [n+2]W^*_{n+1} - e \cdot \tilde{h}_{n+1} \right) \\
&-& s^{-3}a\left( [n+2]W_{n+1} - \tilde{h}_{n+1} \cdot e \right) \\
&-& s^{-1}a^{-1}\left( [n+2]W^*_{n+1} - \tilde{h}_{n+1} \cdot e \right) \\
&+& [n+1]\left( 1-s^{-2} \right) \left( \bar{\beta}_{n+1} -\beta_{n+1} \right) \left( W_n - W^*_n \right) \\
\overset{\ast}{=}&& s^{-1}a\left( [n+1]s^{-1}aW_n + [n+1]s^{-1}\beta_{n+1}a^{-1}W^*_n \right) \\
&+& s^{-3}a^{-1}\left( [n+1]sa^{-1}W^*_n + [n+1]s\bar{\beta}_{n+1}aW_n \right) \\
&-&s^{-3}a\left( [n+1]saW_n + [n+1]s\bar{\beta}_{n+1}a^{-1}W^*_n \right) \\
&-& s^{-1}a^{-1}\left( [n+1]s^{-1}a^{-1}W^*_n + [n+1]s^{-1}\beta_{n+1}aW_n \right) \\
&+& [n+1]\left(1-s^{-2} \right) \left( \bar{\beta}_{n+1} - \beta_{n+1} \right) \left( W_n - W^*_n \right)\\
=&& [n+1]\left( s^{-2}a^2W_n + s^{-2}\beta_{n+1}W^*_n + s^{-2}a^{-2}W^*_n + s^{-2}\bar{\beta}_{n+1}W_n - s^{-2}a^{2}W_n \right. \\
&&\qquad\quad \left. - s^{-2}\bar{\beta}_{n+1}W^*_n - s^{-2}a^{-2}W^*_n - s^{-2}\beta_{n+1}W_n + \bar{\beta}_{n+1}W_n - \bar{\beta}_{n+1}W^*_n - \beta_{n+1}W_n \right. \\
&&\qquad\quad \left. + \beta_{n+1}W^*_n - s^{-2}\bar{\beta}_{n+1}W_n + s^{-2}\bar{\beta}_{n+1}W^*_n + s^{-2}\beta_{n+1}W_n - s^{-2}\beta_{n+1}W^*_n \right) \\
=&& [n+1]\left( \bar{\beta}_{n+1} - \beta_{n+1} \right) \left( W_n - W^*_n \right) \\
=&& \left( \bar{\beta}_{n+1} - \beta_{n+1} \right) \left( \left( [n+1]W_n \right) - \left( [n+1]W^*_n \right) \right) \\
\overset{\ast}{=}&& \left( \bar{\beta}_{n+1} - \beta_{n+1} \right) \left( \left( e \cdot \tilde{h}_{n} + [n]s^{-1}aW_{n-1} + [n]s^{-1}\beta_{n}a^{-1}W^*_{n-1} \right) \right. \\
&& \qquad\qquad\qquad\qquad \left. -\left( e \cdot \tilde{h}_{n} + [n]sa^{-1}W^*_{n-1} + [n]s\bar{\beta}_{n}aW_{n-1} \right) \right) \\
=&& \left( \bar{\beta}_{n+1} - \beta_{n+1} \right) \left( [n]\left( s^{-1} - s\bar{\beta}_{n} \right) aW_{n-1} - [n]\left( s-s^{-1}\beta_{n} \right) a^{-1}W^*_{n-1} \right) \\
=&& [n]\left(\bar{\beta}_{n+1} - \beta_{n+1} \right) \left( s - s^{-1}\beta_{n} \right) \left( aW_{n-1} - a^{-1}W^*_{n-1} \right) \\
=&-& \{n\}\left( aW_{n-1} - a^{-1}W^*_{n-1} \right).
\end{eqnarray*}
Where the last equality follows from a quick computation in the base ring. This completes the proof.
\end{proof}

This next theorem follows directly from Equation \eqref{eq:skewcommutator}, which makes it equivalent to Theorem \ref{thm:powersumcommutator} in some sense. This expresses the left $\cd(A)$-action on $\ca_\cd$ in terms of the right action, and vice versa. This implies a commutation relation for the closures of the BMW symmetrizers. 

\begin{theorem} \label{prop:hncommutator}
For any $n \geq 1$, the relations
\begin{equation}
\tilde{h}_n \cdot e = \sum_{i=0}^n d_i (e \cdot \tilde{h}_{n-i})
\end{equation}
and
\begin{equation}
e \cdot \tilde{h}_n = \sum_{i=0}^n \bar{d}_i (\tilde{h}_{n-i} \cdot e)
\end{equation}
hold in $\ca_\cd$, where
\begin{align*}
d_0 & = 1, \\
d_i & = \sum_{l=0}^{i-1} (1 - s^2) s^{2l-i} a^{i-2l} + (1 - s^{-2}) s^{i-2l} a^{2l-i} \qquad \forall i \geq 1, \\
\bar{d}_i & = \sum_{l=0}^{i-1} (1 - s^{-2}) s^{i-2l} a^{i-2l} + (1 - s^{2}) s^{2l-i} a^{2l-i} \qquad \forall i \geq 1.
\end{align*}
This implies the commutation relations
\begin{equation}
e \cdot \tilde{h}_n - \tilde{h}_n \cdot e = \sum_{i=1}^n \bar{d}_i (\tilde{h}_{n-i} \cdot e)
\end{equation}
or 
\begin{equation}
e \cdot \tilde{h}_n - \tilde{h}_n \cdot e = \sum_{i=1}^n - d_i (e \cdot \tilde{h}_{n-i}).
\end{equation}
\end{theorem}
\begin{proof}
The formula for the $d_i$ were discovered experimentally by coding a solver using the SymPy package in Python. The second equation is just the mirror map applied to the first equation, so we will just prove the first equation.

The idea of the proof depends on a reformulation of Equation \eqref{eq:skewcommutator} as
\[
\tilde{h}_n \cdot e = e \cdot \tilde{h}_n - ( s a + s^{-1} a^{-1} ) ( e \cdot \tilde{h}_{n-1} ) + e \cdot \tilde{h}_{n-2} + ( s^{-1} a + s a^{-1} ) \tilde{h}_{n-1} \cdot e - \tilde{h}_{n-2} \cdot e
\]
and a recursive application of this formula to its last two terms on the right-hand side of the equation. 

The case of $n=0$ is trivial. For $n=1$, just apply the Kauffman skein relation. Now assume the induction hypothesis, that the formula in the statement is true for all $k \leq n-1$. Then apply this assumption to Equation \eqref{eq:skewcommutator}:
\begin{align*}
\tilde{h}_n \cdot e &= e \cdot \tilde{h}_n - ( s a + s^{-1} a^{-1} ) ( e \cdot \tilde{h}_{n-1} ) + e \cdot \tilde{h}_{n-2} + ( s^{-1} a + s a^{-1} ) ( \tilde{h}_{n-1} \cdot e ) - \tilde{h}_{n-2} \cdot e \\
&= e \cdot \tilde{h}_n - ( s a + s^{-1} a^{-1} ) ( e \cdot \tilde{h}_{n-1} ) + e \cdot \tilde{h}_{n-2} + ( s^{-1} a + s a^{-1} ) \sum_{i=0}^{n-1} d_i (e \cdot \tilde{h}_{n-1-i}) \\
&\qquad\qquad\qquad\qquad\qquad\qquad\qquad\qquad\qquad\qquad\qquad\qquad\qquad\qquad - \sum_{i=0}^{n-2} d_i (e \cdot \tilde{h}_{n-2-i}) \\
&= e \cdot \tilde{h}_n + d_1 ( e \cdot \tilde{h}_{n-1} ) + ( s^{-1} a + s a^{-1} ) \sum_{i=1}^{n-1} d_i (e \cdot \tilde{h}_{n-1-i}) - \sum_{i=1}^{n-2} d_i (e \cdot \tilde{h}_{n-2-i}) \\
&= e \cdot \tilde{h}_n + d_1 ( e \cdot \tilde{h}_{n-1} ) + ( s^{-1} a + s a^{-1} ) \sum_{i=0}^{n-2} d_{i+1} (e \cdot \tilde{h}_{n-2-i}) - \sum_{i=1}^{n-2} d_i (e \cdot \tilde{h}_{n-2-i}) \\
&= e \cdot \tilde{h}_n + d_1 ( e \cdot \tilde{h}_{n-1} ) + ( s^{-1} a + s a^{-1} ) d_1 ( e \cdot \tilde{h}_{n-2} ) \\
&\qquad\qquad\qquad\qquad\qquad\qquad\qquad\qquad\quad\quad+ \sum_{i=1}^{n-2} \big( ( s^{-1} a + s a^{-1} ) d_{i+1} - d_i \big) (e \cdot \tilde{h}_{n-2-i}). \\
\end{align*}
It is a straightforward computation to show that $( s^{-1} a + s a^{-1} ) d_1 = d_2$:
\begin{align*}
( s^{-1} a + s a^{-1} ) d_1 &= ( s^{-1} a + s a^{-1} ) \big( ( 1 - s^2 ) s^{-1} a + ( 1 + s^{-2} ) s a^{-1} \big) \\
&= ( 1 - s^2 ) s^{-2} a^2 + (1 - s^{-2} ) s^0 a^0 + ( 1 - s^2 ) s^0 a^0 + ( 1 - s^{-2} ) s^2 a^{-2} \\
&= d_2.
\end{align*}
It's slightly more tedious to show that $( s^{-1} a + s a^{-1} ) d_{i+1} - d_i = d_{i+2}$ for all $i \geq 1$:
\begin{eqnarray*}
&&( s^{-1} a + s a^{-1} ) d_{i+1} - d_i \\
=&& ( s^{-1} a + s a^{-1} ) \sum_{l=0}^{i} (1 - s^2) s^{2l-i} a^{i-2l} + (1 - s^{-2}) s^{i-2l} a^{2l-i} \\
&-& \sum_{l=0}^{i-1} (1 - s^2) s^{2l-(i-1)} a^{(i-1)-2l} + (1 - s^{-2}) s^{(i-1)-2l} a^{2l-(i-1)} \\
=&& \sum_{l=0}^{i} (1 - s^2) s^{2l-(i+1)} a^{(i+1)-2l} + (1 - s^{-2}) s^{(i+1)-2l} a^{2l-(i+1)} \\
&+& \sum_{l=0}^{i} (1 - s^2) s^{2l-(i-1)} a^{(i-1)-2l} + (1 - s^{-2}) s^{(i-1)-2l} a^{2l-(i-1)} \\
&-& \sum_{l=0}^{i-1} (1 - s^2) s^{2l-(i-1)} a^{(i-1)-2l} + (1 - s^{-2}) s^{(i-1)-2l} a^{2l-(i-1)} \\
=&& \sum_{l=0}^{i} (1 - s^2) s^{2l-(i+1)} a^{(i+1)-2l} + (1 - s^{-2}) s^{(i+1)-2l} a^{2l-(i+1)} \\
&+& (1 - s^2) s^{i+1} a^{-1-i} + (1 - s^{-2}) s^{-1-i} a^{i+1} \\
=&& d_{i+2}.
\end{eqnarray*}
This completes the proof of the statement. 
\end{proof}

\begin{remark}
There exists an algebra homomorphism from $\cd(A)$ to the ring of symmetric functions $\Lambda_R$ (see \AP{chapter whatever}). Conjecturally, this map is an isomorphism, which would imply that the sets $\{ \tilde{h}_\lambda \cdot a^i \}_{\lambda, i}$ and $\{ a^i \cdot \tilde{h}_\lambda \}_{\lambda, i}$ over integers $i$ and partitions $\lambda$, where $\tilde{h}_\lambda := \tilde{h}_{\lambda_1} \cdots \tilde{h}_{\lambda_r}$, form bases of $\ca_\cd = \cd(A)[a, a^{-1}]$ \AP{or is this already known separately somehow?}. If so, then Theorem \ref{prop:hncommutator} provides transition formulas between these two bases. 
\end{remark}

As an aside, one might expect similar formulas to hold in the HOMFLYPT case. To our knowledge, there is no HOMFLYPT analogue of Theorem \ref{prop:hncommutator} written down in the literature. Let's do that here. 

\begin{lemma} \label{lem:homfly1}
For all integers $n$, the following relation holds in $\ca_\ch$
\begin{equation}
e \cdot h_n - h_n \cdot e = s a \cdot h_{n-1} - h_{n-1} \cdot  s^{-1} a
\end{equation}
where we use the convention $h_0 = 1$ and $h_n = 0$ if $n < 0$. 
\end{lemma}
\begin{proof}
Recall the power sum elements $P_k$ satisfy the power series equation
\begin{equation} \label{def:Pk}
\sum_{k=1}^\infty \frac{P_k}{k} x^k = \ln \Big( \sum_{n=0}^\infty h_n x^n \Big)
\end{equation}
By \cite[Theorem 4.2]{Mor02b}\AP{fix}, the power sum elements satisfy a commutation relation in $\ca_\ch$
\begin{equation}
e \cdot P_k - P_k \cdot e = (s^{k} - s^{-k}) a^k
\end{equation} 
which may be rephrased as a power series equation 
\[
e \cdot \Big( \sum_{k=1}^\infty P_k x^k \Big) - \Big( \sum_{k=1}^\infty P_k x^k \Big) \cdot e = \sum_{k=1}^\infty s^k a^k - \sum_{k=1}^\infty s^{-k} a^k.
\]
On the left-hand side, use the defining equation \eqref{def:Pk}. Use the power series formulation of natual log on the right-hand side. So we have
\[
\ln \Bigg( e \cdot \Big( \sum_{k=0}^\infty h_k x^k \Big) \Bigg) - \ln \Bigg( \Big( \sum_{k=0}^\infty h_k x^k \Big) \cdot e \Bigg) = \ln ( 1 - s a x ) - \ln ( 1 - s^{-1} a x ).
\]
After moving terms around, using properties of natural log, and exponentiating both sides, we arrive at the equation
\[
\Big( \sum_{n=0}^\infty (h_n \cdot e ) x^n \Big) ( 1 - s a x ) = \Big( \sum_{n=0}^\infty ( e \cdot h_n ) x^k \Big) ( 1 - s^{-1} a x )
\]
which implies the statement of the lemma.
\end{proof}

Recall that the algebra $\ca_\ch$ is equal to the Laurent polynomial ring $\ch(A)^+[a, a^{-1}]$. Under the isomorphism between $\ch(A)^+$ and the ring of symmetric functions $\Lambda_R$, the $h_n$ identify with the complete homogeneous symmetric functions. It is well-known that ordered monomials in the complete homogeneous symmetric functions form a basis of $\Lambda$, hence the sets $\{h_\lambda \cdot a^i \}_{\lambda, i}$ and $\{a^i \cdot h_\lambda \}_{\lambda, i}$ over integers $i$ and partitions $\lambda$, where $h_\lambda := h_{\lambda_1} \cdots h_{\lambda_r}$, form bases of $\ca_\ch$. The following theorem gives transition formulas between these two bases. 

\begin{theorem} \label{prop:homfly2}
The symmetrizer elements $h_n$ satisfy the equations
\[
h_n \cdot e = e \cdot h_n + ( 1 - s^2 ) \sum_{l=1}^{n} s^{-l} ( a^l \cdot h_{n-l} )
\]
and
\[
e \cdot h_n = h_n \cdot e + (1 - s^{-2} ) \sum_{l=1}^{n} s^l ( h_{n-l} \cdot a^l ).
\]
\end{theorem}
\begin{proof}
We will prove the first equality. The second is completely analagous. Proceed by induction. When $n=1$, the statement follows from the HOMFLY skein relation. 

We can rearrange the terms of Lemma \ref{lem:homfly1} to get
\begin{equation} \label{eq:homfly1b}
h_n \cdot e = e \cdot h_n + s^{-1} a ( h_{n-1} \cdot e ) - s a ( e \cdot h_{n-1} ).
\end{equation}
By the induction hypothesis,
\begin{align*}
h_n \cdot e & = e \cdot h_n + s^{-1} a ( h_{n-1} \cdot e ) - s a ( e \cdot h_{n-1} ) \\
& = e \cdot h_n + s^{-1} a \Big( e \cdot h_{n-1} + ( 1 - s^2 ) \sum_{j=1}^{n-1} s^{-j} a^j ( e \cdot h_{n-1-j} ) \Big) - s a ( e \cdot h_{n-1} ) \\
& = e \cdot h_n + ( s^{-1} - s ) a ( e \cdot h_{n-1} ) + ( 1 - s^2 ) \sum_{j=1}^{n-1} s^{-j-1} a^{j+1} ( e \cdot h_{n-1-j} ) \\ 
& = e \cdot h_n + ( 1 - s^2 ) s^{-1} a ( e \cdot h_{n-1} ) + ( 1 - s^2 ) \sum_{j=1}^{n-1} s^{-(j+1)} a^{j+1} ( e \cdot h_{n-(j+1)} ) \\
&= e \cdot h_n + ( 1 - s^2 ) \sum_{l=1}^{n} s^{-l} a^{l} ( e \cdot h_{n-l} )
\end{align*}
where the last equality follows from the substitution $j=l+1$. 
\end{proof}

\section{All Relations}




\section{Perpendicular Relations}




\section{Main Theorem}




\section{Compatibility With the Kauffman Bracket Skein Algebra of the Torus}